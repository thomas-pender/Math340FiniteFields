\documentclass[a4paper,11pt]{article}

\usepackage[utf8]{inputenc}
\usepackage[english]{babel}
\usepackage{amssymb, amsmath, amsthm, mathrsfs}
\usepackage[left=1.0in,right=1.0in,top=1.0in,bottom=1.0in]{geometry}
\usepackage[T1]{fontenc}
\usepackage{array}
\usepackage{longtable}
\usepackage{multirow}
\usepackage{calc}
\usepackage[inline,shortlabels]{enumitem}
\usepackage{changepage}
\usepackage{booktabs}
\usepackage{capt-of}
\usepackage{subcaption}
\usepackage[leftcaption]{sidecap}
\usepackage[numbers]{natbib}
\usepackage{times}
\usepackage{titlesec}
\usepackage{xcolor}
\usepackage{lineno}
\usepackage{xpatch}
\xpatchcmd\swappedhead{~}{.~}{}{}
\allowdisplaybreaks

\newtheoremstyle{mythm}
{}                % Space above
{}                % Space below
{\itshape}        % Theorem body font % (default is "\upshape")
{1.5em}                % Indent amount
{\scshape}       % Theorem head font % (default is \mdseries)
{.}               % Punctuation after theorem head % default: no punctuation
{0.5em}               % Space after theorem head
{}                % Theorem head spec
\theoremstyle{mythm}


\newtheorem*{theorem*}{Theorem}
\newtheorem{theorem}{Theorem}
\newtheorem{fact}[theorem]{Fact}
\newtheorem{proposition}[theorem]{Proposition}
\newtheorem{lemma}[theorem]{Lemma}
\newtheorem{corollary}[theorem]{Corollary}
\newtheorem{question}[theorem]{Question}
\newtheorem{result}[theorem]{Result}
\newtheorem{observation}[theorem]{Observation}
\newtheorem{conjecture}[theorem]{Conjecture}

\newtheoremstyle{mydef}
{}                % Space above
{}                % Space below
{}        % Theorem body font % (default is "\upshape")
{1.5em}                % Indent amount
{\scshape}       % Theorem head font % (default is \mdseries)
{.}               % Punctuation after theorem head % default: no punctuation
{0.5em}               % Space after theorem head
{}                % Theorem head spec
\theoremstyle{mydef}

\newtheorem{example}[theorem]{Example}
\newtheorem{definition}[theorem]{Definition}
\newtheorem{remark}[theorem]{Remark}
\newtheorem*{remark*}{Remark}

\makeatletter
\renewenvironment{proof}[1][\proofname]{\par
  \pushQED{\qed}%
  \normalfont \topsep6\p@\@plus6\p@\relax
  \trivlist
\item\relax
  {\hspace{1.5em}\itshape
    #1\@addpunct{.}}\hspace\labelsep\ignorespaces
}{%
  \popQED\endtrivlist\@endpefalse
}
\makeatother

\def\Box{\hskip1ex\vbox{\hrule height0.6pt\hbox{%
      \vrule height1.3ex width0.6pt\hskip0.8ex
      \vrule width0.6pt}\hrule height0.6pt
  }}
\renewcommand{\qed}{\Box}

\newcommand{\red}[1]{\textcolor{red}{#1}}
\newcommand{\blue}[1]{\textcolor{blue}{#1}}
\newcommand{\purple}[1]{\textcolor{magenta}{#1}}
\newcommand{\ddet}{\text{det}}
\renewcommand{\pmod}[1]{\text{ (mod $#1$)}}
\newcommand{\mmod}[2]{#1\text{ mod }#2}
\newcommand{\abs}[1]{\left\vert #1 \right\vert}
\newcommand{\C}{\mathbb{C}}
\newcommand{\Z}{\mathbb{Z}}
\newcommand{\LL}{\mathscr{G}}
\newcommand{\z}{\mathbin{\ooalign{$\hidewidth i \hidewidth$\cr$\phantom{+}$}}}
\newcommand{\y}{\mathbin{\ooalign{$\hidewidth j \hidewidth$\cr$\phantom{+}$}}}
\newcommand{\gf}{\text{GF}}

\newcolumntype{R}{>{\scriptsize}r}
\newcolumntype{L}{>{\scriptsize}l}
\newcolumntype{C}{>{\scriptsize}c}

\renewcommand{\citenumfont}[1]{\textbf{#1}}
\renewcommand{\bibnumfmt}[1]{\textbf{#1.}}

\titleformat{\section}{\normalfont\Large\bfseries\centering}{\thesection.}{0.5em}{}
\titleformat{\subsection}{\normalfont\bfseries}{\thesubsection.}{0.5em}{}

\newenvironment{myabstract}{\vspace{1em}\begin{adjustwidth}{3em}{3em}\begin{small}\textbf{Abstract.}}{\end{small}\end{adjustwidth}\vspace{1em}}
\newenvironment{mykeywords}{\vspace{1em}\begin{adjustwidth}{3em}{3em}\begin{small}\textbf{Keywords.}}{\end{small}\end{adjustwidth}\vspace{1em}}

\DeclareCaptionLabelSeparator{custom}{.}
\DeclareCaptionLabelFormat{custom}
{%
  \textsc{#1 #2}
}
\DeclareCaptionFormat{custom}
{%
  #1#2 #3
}
\captionsetup
{
  format=custom,%
  labelformat=custom,%
  labelsep=custom
}

\begin{document}

\begin{center}
  {\Large\bfseries Math 340 Tutorial \\ November 24$^{\text{th}}$}
\end{center}

\noindent{\bf Question 1.} Let $K$ be a finite degree extension of a finite
field $F$. Show there exists and element $a \in K$ for which $K=F(a)$. \\

\blue{
  We know that there exists some $\alpha \in K^*$ for which $K^*=\langle \alpha
  \rangle$. But then $K=F(\alpha)$.
} \\

\noindent{\bf Question 2.} How many primitive elements of $GF(81)$ are there? Of
$\gf(32)$? \\

\blue{
  We have the number of primitive elements of $\gf(81)$ is $\phi(80) =
  \phi(2^45) = 2^34 = 32$. The number of primitive elements of $\gf(32)$ is
  $\phi(31) = 30$.
} \\

\noindent{\bf Question 3.} Determine the finite fields whose largest proper
subfield is $\gf(2^5)$. \\

\blue{$\gf(2^{10})$.} \\

\noindent{\bf Question 4.} Let $\alpha,\beta \in \gf(81)$ with $\abs{\alpha}=5$
and $\abs{\beta}=16$. Show that $\alpha\beta$ is a primitive element. \\

\blue{
  Since $(5,16) = 1$, we have $\langle \alpha \rangle \times \langle \beta
  \rangle = \{1\}$ and hence $\abs{\langle \alpha\beta \rangle} = 5\cdot16=80$.
} \\

\noindent{\bf Question 5.} Let $p$ be an odd prime, and let $a \in \gf(p)$ be
a nonsquare. Show that $a$ is a square in $\gf(p^n)$ if $n$ is even, and $a$ is
a nonsquare in $\gf(p^n)$ if $n$ is odd. \\

\blue{
  Let $g$ be a primitive element of $\gf(p^n)$, and let $v = \frac{q^n-1}{q-1} =
  q^{n-1}+q^{n-2}+\cdots+q+1$. Then $g^v$ is a primitive element of $\gf(p)$.
  Since $a \in \gf(p)$, $a \neq 0$, we have there is some $k \in
  \{1,3,\dots,p-2\}$ for which $a=g^{vk}$; in particular, $k$ is odd by our
  assumption on $a$. The parity of $v$ equals the parity of $n$. Hence, $vk$ is
  even if and only if $n$ is even.
} \\

Define $f \in F[x]$ by $f(x)=x^n-1$. The roots of $f$ are the $n$-roots of unity
over $F$, and the splitting field $F^{(n)}$ of $f$ over $F$ is the $n$-th
cyclotomic field (over $F$). Use $E^{(n)}$ to denote the roots of $f$. \\

\noindent{\bf Question 6.} Suppose that $\text{char}(F)=p$, a prime. Show:
\begin{enumerate*}[{\bf(a)}]
\item If $(p,n)=1$, then $E^{(n)}$ is a multiplicative cyclic group of order
  $n$.
\item If $p \mid n$, write $n=mp^e$ with $(p,m)=1$. Then $F^{(m)}=F^{(n)}$ and
  $E^{(m)}=E^{(n)}$, and the roots of $f$ in $F^{(n)}$ are the elements of
  $E^{(m)}$ each with multiplicity $p^e$.
\end{enumerate*}

\blue{
  \begin{enumerate}[{\bf (a)}]
  \item We have $E^{(n)} \neq \emptyset$ as $1 \in E^{(n)}$. If $a,b \in
    E^{(n)}$, then $(ab^{-1})^n = a^n(b^n)^{-1}=1\cdot1=1$; hence, $ab^{-1} \in
    E^{(n)}$.
  \item This is clear because $x^n-1 = x^{mp^e}-1 = (x^m-1)^{p^e}$; so, the
    result follows from part (a).
  \end{enumerate}
}

Suppose that $(p,n)=1$, and let $\zeta$ be a generator of $E^{(n)}$. The
polynomial
\[
  Q_n(x) = \prod_{\begin{smallmatrix}s=1 \\ (s,n)=1 \end{smallmatrix}}^n(x-\zeta^s)
\]

\noindent is called the $n$-th cyclotomic polynomial over $F$. \\

\noindent{\bf Question 7.} Show:
\begin{enumerate}[{\bf (a)}]
\item $x^n-1 = \prod_{d \mid n}Q_d(x)$.

\item $Q_n(x) = \prod_{d \mid n}(x-1)^{\mu(n/d)}$.

\item $Q_{p^k}(x)=1+x^{p^{k-1}}+x^{2p^{k-1}}+\cdots+x^{(p-1)p^{k-1}}$.

\item If $F=\gf(q)$ with $(q,n)=1$, then $Q_n$ factors into $\phi(n)/d$ distinct
  monic irreducible polynomials in $F[x]$ of the same degree $d$, $F^{(n)}$ is
  the splitting field of any such factor over $F$, and $[F^{(n)}:F]=d$, where
  $d$ is multiplicative order of $q$ modulo $n$.
\end{enumerate}

\blue{
  \begin{enumerate}[{\bf (a)}]
  \item Each $n$-th root of unity is a primitive $d$-th root of unity for
    exactly one divisor $d$ of $n$. Explicitly, if $\zeta$ is a primitive $n$-th
    root of unity, and if $\zeta^s$ is an arbitrary $n$-root of unity, then
    $d=n/(s,n)$. Since
    \[
      Q_n(x) = \prod_{s=0}^{n-1}(x-\zeta^s),
    \]
    the result follows by collecting those factors $x-\zeta^s$ which are
    primitive $d$-th roots of unity.
  \item Apply the multiplicative version of M\"obius inversion to part (a).
  \item By induction on $k$. If $k=1$, the result is clear. Assume it is true
    for $k\geqq1$. Then $Q_{r^{k+1}} =
    \frac{x^{r^{k+1}}-1}{\prod_{s=0}^kQ_{r^s}(x)} =
    \frac{x^{r^{k+1}}-1}{x^{r^k}-1}$.
  \item If $\zeta$ is a primitive $n$-th root of unity, then $F^{(n)}$ is the
    algebraic extension $F(\zeta)$. Observe that $\zeta \in \gf(q^k)$ if and
    only if $\zeta^{q^{k-1}}-1=0$ if nad only if $q^k \equiv 1 \pmod{n}$. The
    smallest $k$ for which this holds is $k=d$. So, $\zeta \in \gf(q^d)$ but no
    proper subfield thereof. Thus, the minimal polynomial of $\zeta$ has degree
    $d$, and since $\zeta$ was an arbitrary root of $Q_n(x)$, the result follows.
  \end{enumerate}
}

\noindent{\bf Question 8.} Let $f \in \gf(q)[x]$ have degree $m$ with $f(0) \neq
0$. Show there exists a positive integer $e \leqq q^m-1$ such that $f$ divides
$x^e-1$. \\

\blue{
  The ring $\gf(q)[x]/(f)$ has order $q^m$. Therefore, among the residues
  $x^k+(f)$, $k=0,\dots,q^m-1$, there must be $a<b$ for which $x^a \equiv
  x^b \pmod{f}$. Since $(x,f)=1$, we have that $x^{a-b} \equiv 1 \pmod{f}$.
  Therefore, $f \mid x^{a-b}-1$ and $0< a-b \leqq q^m-1$.
} \\

For a polynomial $f \in \gf(q)[x]$ with $f(0) \neq 0$, the order $\text{ord}(f)$
of $f$ is the smallest positive integer $e$ for which $f \mid x^e-1$. If $x \mid
f$, write $f=x^ag$ with $g(0) \neq 0$. We then define $\text{ord}(f) \equiv
\text{ord}(g)$. \\

\noindent{\bf Question 9.} Let $f \in \gf(q)[x]$ be irreducible of degree $m$
with $f(0) \neq 0$. Show that $\text{ord}(f)$ is the multiplicative order of any
one of its roots in $\gf(q^m)$. Show additionally that $\text{ord}(f) \mid
q^m-1$. \\

\blue{
  $\gf(q^m)$ is the splitting field of $f$. The roots of $f$ have the same
  order in $\gf(q^m)^*$. But $\alpha^e=1$ if and only if $f \mid x^e-1$. The
  result now follows.
} \\

\noindent{\bf Question 10.} Show the number of monic irreducible polynomials in
$\gf(q)[x]$ of degree $m$ and order $e$ is $\phi(e)/m$ if $e \geqq 2$ and $m$ is
the multiplicative order of $q$ modulo $e$, equal to 2 if $m=e=1$, and equal to
0 in all other cases. \\

\blue{
  Let $f \in \gf(q)[x]$ be irreducible with $f(0) \neq 0$. Then
  $\text{ord}(f)=e$ if and only if every root of $f$ is a primitive $e$-th root
  of unity over $\gf(q)$ if and only if $f \mid Q_e(x)$. We've shown every monic
  irreducible factor of $Q_e(x)$ has the same degree $m$, the least positive
  integer such that $q^m \equiv 1 \pmod{e}$. The number of such factors must
  then be $\phi(e)/m$. For $m=e=1$ we also have to take into account $f(x)=x$.
} \\

\noindent{\bf Question 11.} Show that $f \in \gf(q)[x]$ is an irreducible factor
of $x^{q^n}-x$ if and only if $\text{deg}(f) \mid n$. Moreover, show that the
product of all irreducible polynomials whose degree divides $n$ is equal to
$x^{q^n}-x$. \\

\blue{
  Let $f$ be an irreducible divisor of $x^{q^n}-x$, and let $\alpha$ be a root
  of $f$. Then $\alpha \in \gf(q^n)$ whence $\gf(q)(\alpha)$ is a subfield of
  $\gf(q^n)$. But then $[\gf(q)(\alpha) : \gf(q)]=m$ divides $[\gf(q^n) :
  \gf(q)] = n$.
}

\blue{
  Conversely, if $m$ divides $n$, then $\gf(q^m) \subseteqq \gf(q^n)$.
  Furthermore, if $\alpha$ is a root of $f$, then $\gf(q)(\alpha)=\gf(q^m)$;
  hence, $\alpha \in \gf(q^n)$ and so $f \mid x^{q^n}-x$.
}

\blue{
  We have shown that the monic irreducible factors $f$ of $x^{q^n}-x$ are
  exactly those whose degrees divide $n$. We know $x^{q^n}-x$ has no repeated
  roots, so every irreducible factor of $x^{q^n}-x$ appears exactly once.
} \\

\noindent{\bf Question 12.} Let $N_q(n)$ be the number of irreducible
polynomials over $\gf(q)$ of degree $n$. Show that
\[
  N_q(n) = \sum_{d \mid n}\mu(n)q^{n/d}.
\]

\blue{
  The previous question implies $q^n = \sum_{d \mid n}dN_d(q)$. The result now
  follows from M\"obius inversion.
} \\

\noindent{\bf Question 13.} Let $I(q,n,x)$ be the product of all irreducible
polynomials in $\gf(q)[x]$ of degree $n$. Show that
\[
  I(q,n,x) = \prod_{d \mid n}(x^{q^d}-x)^{\mu(n/d)} = \prod Q_m(x)
\]

\noindent where the product is extended over all divisor $m$ of $q^n-1$ such
that $n$ is multiplicative order of $q$ modulo $m$. \\

\blue{
  We've observed already that $x^{q^n}-x = \prod_{d \mid n}I(q,d,x)$; so, the
  first identity follows from M\"obius inversion. Next, let $S \subseteqq
  \gf(q^n)$ be the set of elements of degree $n$. Thus each $\alpha \in S$ is a
  root of $I(q,n,x)$. On the other hand, if $\beta$ is a root of $I(q,n,x)$,
  then $\beta$ is a root of some monic irreducible of degree $n$, hence $\beta
  \in S$. We therefore have
  \[
    I(q,n,x) = \prod_{\alpha \in S}(x-\alpha).
  \]
  Now $\text{ord}(\alpha)=m$, $\alpha \in S$, is such that $n=\text{ord}_m(q)$.
  For such a divisor $m$ of $q^n-1$, let $S_m \subseteqq S$ be the subset of
  elements of $S$ of order $m$. Then $S$ is the disjoint union of the $S_m$,
  hence
  \[
    I(q,n,x) = \prod_m\prod_{\alpha \in S_m}(x-\alpha).
  \]
  Now $S_m$ contains all the elements of $\gf(q^n)^*$ of order $m$, i.e., all
  the primitive $m$-th roots of unity over $\gf(q)$. It follows that
  \[
    \prod_{\alpha \in S_m}(x-\alpha) = Q_m(x).
  \]
}

\end{document}

